\documentclass[11pt,fleqn]{article}
\usepackage{../cs70}
\lecture{23}
\def\title{Note \the\lecturenumber}

\begin{document}
\maketitle

\section*{How to Lie with Statistics}

``There are three kinds of lies: lies, damned lies, and 
statistics,'' as Mark Twain is reputed to have said. 
In this lecture we will see why 
statistics are so easy to misuse, and some of the ways 
we must be careful while evaluating statistical claims. 

First a very simple but important point about statistics.
Statistics cannot be used to establish causation,
they can only show correlations. As an example, the 
governor of a certain state is concerned about the 
test scores of high school students in the state. 
One of his aides brings up an interesting statistic:
there is a very strong link between student test scores
and the taxes paid by the parents of the student. 
The parents of high scoring students pay more taxes. 
The aide's suggestion for increasing student test 
scores is unusual; sharply increase tax rates. Surely
student test scores will follow! The fallacy, of 
course, is that even though there is a correlation 
between test scores and parents taxes, there is likely 
no causal connection. A better explanation is that there
is some hidden variable that explains the correlation. 
In this case the obvious choice is the income of the 
parents. This determines the taxes paid. And since 
the quality of high school that a student attends is 
to a large extent determined by the parent's income,
we see a causal link from parent's income to both 
taxes and test scores. 

Let us now turn to a very important paradox in 
probability called Simpson's paradox, described by
Simpson in his 1951 paper. Let us start with an example, 
which studies the 20 year survival rate of smokers.
A paper by Appleton, French, and Vanderpump
(1996, \emph{American Statistician}) surveyed 1314 English women in
1972--1974 and again after 20 years. Their results are summarized
in the following table:

\begin{center}
\begin{tabular}{|c|c|c|c|c|}
  \hline
  % after \\: \hline or \cline{col1-col2} \cline{col3-col4} ...
  Smoker & Dead & Alive & Total & \% Dead \\ \hline
  Yes & 139 & 443 & 582 & 24 \\
  No & 230 & 502 & 732 & 31 \\\hline
  Total & 369 & 945 & 1314 & 28 \\
  \hline
\end{tabular}
\end{center}

From this data one might conclude that non-smoking kills!
How does one explain this unusual data? The answer lies
in the composition of the two groups.
Smoking was unpopular in the middle of the 20th century
among women in England, and it increased only in the 1970's;
but it was mostly young women who started to smoke. 
Therefore, the sample of smokers in the study was heavily 
biased towards young women, whose expected lifespan was of
course much larger than $20$ years. This becomes clearer when 
we look more closely at the results of the study broken
down by age group:

\begin{center}
\begin{tabular}{|c|c|c|c|c|c|c|c|c|c|c|}
  \hline
  % after \\: \hline or \cline{col1-col2} \cline{col3-col4} ...
  Age group & \multicolumn{2}{|c|}{18--24} & \multicolumn{2}{|c|}{25--34} & \multicolumn{2}{|c|}{35--44} & \multicolumn{2}{|c|}{45--54} & \multicolumn{2}{|c|}{55--54}\\ \hline
  Smoker & Y & N & Y & N & Y & N & Y & N & Y & N \\ \hline \hline
  Dead & 2 & 1 & 3 & 5 & 11 & 7 & 27 & 12 & 51 & 40 \\
  Alive & 53 & 61 & 121 & 152 & 95 & 114 & 103 & 66 & 64 & 81 \\\hline
  Ratio & \multicolumn{2}{|c|}{2.3}& \multicolumn{2}{|c|}{0.75} & \multicolumn{2}{|c|}{2.4}& \multicolumn{2}{|c|}{1.44} & \multicolumn{2}{|c|}{1.61}\\
  \hline
\end{tabular}
\end{center}

The interesting thing to notice is that the fatality rates are 
significantly higher for smokers in almost every age group!
The data could be made even more dramatic by increasing the 
smoking fatalities in the couple of exceptional groups by 
one or two, thereby achieving the following strange result:
in each separate category, the percentage of fatalities among
smokers is higher, and yet the overall percentage of fatalities
among smokers is lower.  
This is an example of a phenomenon known as  \textbf{Simpson's paradox}.
(For a nice treatment of this topic, and a different example, 
you may wish to consult the Wikipedia entry.)

A real word example is provided very close to home by the UC Berkeley 
gender bias case. In 1973, UC Berkeley was sued for bias against women 
applying to grad school.  Data showed that 44\% of men were admitted 
and only 30\% of women. Since admission is decided by departments,
the University decided to investigate which departments were ``discriminating''
against women.  It turned out that none of them were! Here is some
admissions data for the four largest departments:
\begin{center}
\begin{tabular}{|c|c|c|c|c|}
  \hline
  % after \\: \hline or \cline{col1-col2} \cline{col3-col4} ...
  Department & \#male applicants & \#female applicants & \%male admit & \%female admit \\\hline
  A & 825 & 108 & 62 & 82 \\
  B & 560 & 25  & 63 & 68 \\
  C & 325 & 593 & 37 & 34 \\
  D & 417 & 375 & 33 & 35 \\
  \hline
\end{tabular}
\end{center}

The explanation is that women applied in larger numbers to departments 
that had lower admission rates.

\end{document}


