% !TEX TS-program = pdflatex
% !TEX encoding = UTF-8 Unicode

% This is a simple template for a LaTeX document using the "article" class.
% See "book", "report", "letter" for other types of document.

\documentclass[11pt,preview]{standalone} % use larger type; default would be 10pt

\usepackage[utf8]{inputenc} % set input encoding (not needed with XeLaTeX)


\usepackage{../../markup}
%%% Examples of Article customizations
% These packages are optional, depending whether you want the features they provide.
% See the LaTeX Companion or other references for full information.

%%% PAGE DIMENSIONS
\usepackage{geometry} % to change the page dimensions
\geometry{a4paper} % or letterpaper (US) or a5paper or....
% \geometry{margin=2in} % for example, change the margins to 2 inches all round
% \geometry{landscape} % set up the page for landscape
%   read geometry.pdf for detailed page layout information

\usepackage{graphicx} % support the \includegraphics command and options
\usepackage{color}
% \usepackage[parfill]{parskip} % Activate to begin paragraphs with an empty line rather than an indent

%%% PACKAGES
\usepackage{amsmath, amsfonts,amssymb}
\usepackage{booktabs} % for much better looking tables
\usepackage{array} % for better arrays (eg matrices) in maths
\usepackage{paralist} % very flexible & customisable lists (eg. enumerate/itemize, etc.)
\usepackage{verbatim} % adds environment for commenting out blocks of text & for better verbatim
\usepackage{subfig} % make it possible to include more than one captioned figure/table in a single float
% These packages are all incorporated in the memoir class to one degree or another...

%%% HEADERS & FOOTERS
\usepackage{fancyhdr} % This should be set AFTER setting up the page geometry
\pagestyle{fancy} % options: empty , plain , fancy
\renewcommand{\headrulewidth}{0pt} % customise the layout...
\lhead{}\chead{}\rhead{}
\lfoot{}\cfoot{\thepage}\rfoot{}

%%% SECTION TITLE APPEARANCE
\usepackage{sectsty}
\allsectionsfont{\sffamily\mdseries\upshape} % (See the fntguide.pdf for font help)
% (This matches ConTeXt defaults)

%%% ToC (table of contents) APPEARANCE
\usepackage[nottoc,notlof,notlot]{tocbibind} % Put the bibliography in the ToC
\usepackage[titles,subfigure]{tocloft} % Alter the style of the Table of Contents
\renewcommand{\cftsecfont}{\rmfamily\mdseries\upshape}
\renewcommand{\cftsecpagefont}{\rmfamily\mdseries\upshape} % No bold!

\newcommand{\N}{\mathbb{N}}
\newcommand{\Z}{\mathbb{Z}}
\newcommand{\R}{\mathbb{R}}
\newcommand{\Q}{\mathbb{Q}}
%%% END Article customizations

%%% The "real" document content comes below...

%\date{} % Activate to display a given date or no date (if empty),
         % otherwise the current date is printed 

\begin{document}
\config{name}{Infinity, Countability, and Computability}

\noindent{\bf Infinity and Countability}

\begin{enumerate}
	\item Which of the following are true for the cardinality of the set $\{1,2,3\}$?
	\begin{Multi}
	\begin{itemize}
		\TrueChoice\item Finite.
		\TrueChoice\item Countable.
	\end{itemize}
	\Solution This is a finite set since it has only 3 elements. A set is countable if and only if there is a bijection between it and a subset of $\mathbb{N}$. But this is actually a subset of $\mathbb{N}$ itself. So the identity function is the desired bijection. In general any finite set is countable.
	\end{Multi}
	
	\item Which of the following are true for the cardinality of the set of multiples of $5$?
	\begin{Multi}
	\begin{itemize}
		\FalseChoice\item Finite.
		\TrueChoice\item Countable.
	\end{itemize}
	\Solution This set obviously has infinitely many elements since $5k$ for any $k\in\mathbb{N}$ is in it. The function $x\to \frac{x}{5}$ is a bijection between this set and the set of natural numbers. So this is a countable set.
	\end{Multi}


	\item Suppose that a set $S$ is countable. Which of the following are true about $S$?
	\begin{Multi}
	\begin{itemize}
		\TrueChoice\item Any subset of $S$ is also countable.
		\FalseChoice\item Any strict (not equal to $S$ itself) subset of $S$ is finite.
		\TrueChoice\item  If we add an element to $S$ it remains countable.
	\end{itemize}
	\Solution By definition there is a bijection between $S$ and some subset of $\mathbb{N}$. That same bijection restricted to any subset of $S$ is also a bijection between that subset and some subset of $\mathbb{N}$.
	
	It is not true that any strict subset of $S$ is finite. For example the set of natural numbers is countable, but the set of odd numbers is a strict subset which is not finite.
	
	If $S$ is finite, adding an element to it makes it finite. If $S$ is infinite, then there is a bijection between it and $\mathbb{N}$. Now when we add an element to $S$ we can shift everything up by $1$ (so if an element was being mapped to $x$, now it's mapped to $x+1$), and map the new element to the lowest element of $\mathbb{N}$, i.e. $0$.
	\end{Multi}

	\item Which of the following sets are uncountable?
	\begin{Multi}
	\begin{itemize}
		\TrueChoice\item The interval $[0, 1]$.
		\FalseChoice\item The total number of computer programs that could ever be written.
		\TrueChoice\item The interval $[0, 10]$.
		\TrueChoice\item The set of real numbers that have a digit $1$ when written down.
	\end{itemize}
	\Solution The interval $[0,1]$ can be shown to be uncountable using Cantor's diagonalization method.
	
	Every computer program is simply a finite sequence of characters. If we assign each character a natural number, then a computer program is just a finite sequence of natural numbers. These are countable, since there is a bijection between them and the polynomials with natural coefficients which was shown to be a countable set.
	
	There is a bijection between the interval $[0,10]$ and the interval $[0,1]$. Namely $x\to \frac{x}{10}$.
	
	The set of real numbers that have a digit $1$ in their representation obviously includes the set $[10,11]$ which is uncountable.
	\end{Multi}

	\item In what cases does a set $S$ have the same cardinality as the set $S'$ obtained from $S$ by adding an element to it?
	\begin{Multi}
	\begin{itemize}
		\FalseChoice\item If $S$ is finite.
		\TrueChoice\item If $S$ is countable and infinite.
		\TrueChoice\item If $S$ is uncountable and infinite.
	\end{itemize}
	\Solution When $S$ is finite, then $S$ plus one element has strictly larger cardinality.
	
	But when $S$ is infinite, we can extract an infinite sequence out of it by the following method: we pick an element $a_1$; there must be another element $a_2$; there must be another element $a_3$, and so on. This process never stops because $S$ is infinite. Now construct a bijection between $S$ and $S'$ by mapping each $a_i$ to $a_{i+1}$, mapping the new element to $a_1$, and mapping other elements to themselves. This shows that $S$ and $S'$ have the same cardinality.
	\end{Multi}

\end{enumerate}

\noindent{\bf Computability}

\begin{enumerate}
	\item Why can't we solve the Halting problem by just simulating the given program on the given input and seeing if it stops?
	\begin{Multi}
	\begin{itemize}
		\FalseChoice\item Because we cannot simulate programs.
		\TrueChoice\item Because in case the answer is no, our simulation would never stop.
	\end{itemize}
	\Solution We can indeed simulate programs. That is what virtualization softwares that most people are familiar with do.
	
	The reason is that we need to output an answer no matter whether it's Yes or No. But if a program is not going to halt (in which case the answer is No), our simulation also never stops.
	\end{Multi}
	
	\item 

\end{enumerate}

\end{document}
