% !TEX TS-program = pdflatex
% !TEX encoding = UTF-8 Unicode

% This is a simple template for a LaTeX document using the "article" class.
% See "book", "report", "letter" for other types of document.

\documentclass[11pt, preview]{standalone} % use larger type; default would be 10pt

\usepackage[utf8]{inputenc} % set input encoding (not needed with XeLaTeX)

%%% Examples of Article customizations
% These packages are optional, depending whether you want the features they provide.
% See the LaTeX Companion or other references for full information.

%%% PAGE DIMENSIONS
\usepackage{geometry} % to change the page dimensions
\geometry{a4paper} % or letterpaper (US) or a5paper or....
% \geometry{margin=2in} % for example, change the margins to 2 inches all round
% \geometry{landscape} % set up the page for landscape
% read geometry.pdf for detailed page layout information

\usepackage{graphicx} % support the \includegraphics command and options

% \usepackage[parfill]{parskip} % Activate to begin paragraphs with an empty line rather than an indent

%%% PACKAGES
\usepackage{amsmath,amsthm,amsfonts}
\usepackage{ dsfont }
\usepackage{booktabs} % for much better looking tables
\usepackage{array} % for better arrays (eg matrices) in maths
\usepackage{paralist} % very flexible & customisable lists (eg. enumerate/itemize, etc.)
\usepackage{verbatim} % adds environment for commenting out blocks of text & for better verbatim
\usepackage{subfig} % make it possible to include more than one captioned figure/table in a single float
% These packages are all incorporated in the memoir class to one degree or another...

\def\N{\mathbb{N}}

%%% HEADERS & FOOTERS
\usepackage{fancyhdr} % This should be set AFTER setting up the page geometry
\pagestyle{fancy} % options: empty , plain , fancy
\renewcommand{\headrulewidth}{0pt} % customise the layout...
\lhead{}\chead{}\rhead{}
\lfoot{}\cfoot{\thepage}\rfoot{}

%%% SECTION TITLE APPEARANCE
\usepackage{sectsty}
\allsectionsfont{\sffamily\mdseries\upshape} % (See the fntguide.pdf for font help)
% (This matches ConTeXt defaults)

%%% ToC (table of contents) APPEARANCE
\usepackage[nottoc,notlof,notlot]{tocbibind} % Put the bibliography in the ToC
\usepackage[titles,subfigure]{tocloft} % Alter the style of the Table of Contents
\renewcommand{\cftsecfont}{\rmfamily\mdseries\upshape}
\renewcommand{\cftsecpagefont}{\rmfamily\mdseries\upshape} % No bold!

%%% END Article customizations

%%% The "real" document content comes below...

% \date{} % Activate to display a given date or no date (if empty),
% otherwise the current date is printed 

\usepackage{../../markup}

\begin{document}
\config{name}{Propositions and Proofs}
\noindent{\bf Propositions and Proofs}

\hspace{6ex}A proposition is a statement which is either true or
false. 

\hspace{6ex}Are the following propositions?

\begin{enumerate}
  % -------------------------------------------------------
\item $2+2=4$
  \begin{enumerate}
    \begin{Choices}
      \TrueChoice\item Yes, it is a proposition.
      \FalseChoice\item No, it is not a proposition.
      \Solution This is a proposition and it is true.
    \end{Choices}
  \end{enumerate}

  % -------------------------------------------------------
\item $x+2=4$
  \begin{enumerate}
    \begin{Choices}
      \FalseChoice\item Yes, it is a proposition.
      \TrueChoice\item No, it is not a proposition.
      \Solution This is not a proposition. It is a predicate: whether it is true or false is predicated on the value of $x$.
    \end{Choices}
  \end{enumerate}

  % -------------------------------------------------------
\item All photos are taken by some human.
  \begin{enumerate}
    \begin{Choices}
      \TrueChoice\item Yes, it is a proposition.
      \FalseChoice\item No, it is not a proposition.
      \Solution This is a proposition and it is false. Search for ``macaque selfie'' online for a counterexample.
    \end{Choices}
  \end{enumerate}

  % -------------------------------------------------------
\item How is your new semester so far?
  \begin{enumerate}
    \begin{Choices}
      \FalseChoice\item Yes, it is a proposition.
      \TrueChoice\item No, it is not a proposition.
      \Solution This is not a proposition. A question is not a statement.
    \end{Choices}
  \end{enumerate}



\item
Let $\mathbb{X} = \{\text{photos}\}$ and 
$\mathbb{Y} = \{\text{humans}\}$, which one of the 
following is equivalent to 
``All photos are taken by some human''?
\begin{Choices}
  \begin{enumerate}
  \FalseChoice\item $(\forall x \in \mathbb{X})(\forall y \in \mathbb{Y})(x\text{ is taken by }y)$
  \TrueChoice\item $(\forall x \in \mathbb{X})(\exists y \in \mathbb{Y})(x\text{ is taken by }y)$
  \FalseChoice\item $(\exists x \in \mathbb{X})(\forall y \in \mathbb{Y})(x\text{ is taken by }y)$
  \FalseChoice\item $(\exists x \in \mathbb{X})(\exists y \in \mathbb{Y})(x\text{ is taken by }y)$
  \Solution Let's break down this proposition. $(\forall x \in \mathbb{X})(\exists y \in \mathbb{Y})(x \text{ is taken by } y)$ translates to "For all photos $x$, there exists some human $y$ such that the human $y$ took the photo $x$."
  \end{enumerate}
\end{Choices}


\item
Let $\mathbb{Z}$ denote the set of all integers, and let
$P(x)$ denote the proposition formula $x \geq 0$, which ones of the
following are equivalent to ``For every pair of integers, at least one
of them is negative''?
\begin{Multi}
  \begin{enumerate}
    \TrueChoice\item $(\forall x \in \mathbb{Z})(\forall y \in \mathbb{Z})(\lnot P(x) \lor \lnot P(y))$
    \FalseChoice\item $(\forall x \in \mathbb{Z})(\forall y \in \mathbb{Z})\lnot (P(x) \lor P(y))$
    \TrueChoice\item $\lnot ((\exists x \in \mathbb{Z})(\exists y \in \mathbb{Z})(P(x) \land P(y))$
    \TrueChoice\item $(\forall x \in \mathbb{Z})\lnot ((\exists y \in \mathbb{Z})(P(x) \land P(y))$
    \Solution a), c), and d) are all correct. Let's consider them one by one. Choice a) translates to "For all integers $x$ and $y$, either $x$ is less than 0 or $y$ is less than 0 or both are less than 0". Choice c) translates to "It is NOT true that there exist integers $x$ and $y$ such that both $x$ and $y$ are greater than or equal to 0." Choice d) translates to "For every integer $x$, there does not exist an integer $y$ such that both $x$ and $y$ are greater than or equal to 0". Note that d) can be easily derived from c) using the rules for distributing $\lnot$ across quantifiers. 
  \end{enumerate}
\end{Multi}

\item Select the correct truth table for the boolean function
\begin{equation*}
Y = (A \implies \lnot B) \land (C \implies B).
\end{equation*}
\begin{Choices}
  \Hint Note that $P \implies Q$ is logically equivalent to 
  $\lnot P \lor Q$. Try converting $(A \implies \lnot B)$ 
  and $(C \implies B)$ to their equivalent disjuction forms
  first.
  \begin{enumerate}[(a)]
    \FalseChoice\item \begin{tabular}{ c | c | c || c }
      $A$ & $B$ & $C$ & $Y$ \\
      \hline
      0 & 0 & 0 & 1 \\
      0 & 0 & 1 & 0
    \end{tabular}
    \TrueChoice\item \begin{tabular}{ c | c | c || c }
      $A$ & $B$ & $C$ & $Y$ \\
      \hline
      0 & 0 & 0 & 1 \\
      0 & 0 & 1 & 0 \\
      0 & 1 & 0 & 1 \\
      0 & 1 & 1 & 1 \\
      1 & 0 & 0 & 1 \\
      1 & 0 & 1 & 0 \\
      1 & 1 & 0 & 0 \\
      1 & 1 & 1 & 0 
    \end{tabular}
    \FalseChoice\item \begin{tabular}{ c | c | c || c }
      $A$ & $B$ & $C$ & $Y$ \\
      \hline
      0 & 0 & 0 & 0 \\
      0 & 0 & 1 & 1 \\
      0 & 1 & 0 & 0 \\
      0 & 1 & 1 & 0 \\
      1 & 0 & 0 & 0 \\
      1 & 0 & 1 & 1 \\
      1 & 1 & 0 & 1 \\
      1 & 1 & 1 & 1 
    \end{tabular}
    \FalseChoice\item \begin{tabular}{ c | c | c || c }
      $A$ & $B$ & $C$ & $Y$ \\
      \hline
      0 & 0 & 0 & 1 \\
      0 & 0 & 1 & 1 \\
      0 & 1 & 0 & 1 \\
      0 & 1 & 1 & 1 \\
      1 & 0 & 0 & 1 \\
      1 & 0 & 1 & 1 \\
      1 & 1 & 0 & 1 \\
      1 & 1 & 1 & 1 
    \end{tabular}
    \Solution Let's build up the truth table in stages, by first finding truth tables for $A \implies \lnot B$ and $C \implies B$. Recall that $P \implies Q$ is logically equivalent to $\lnot P \lor Q$. This gives us the following truth tables for $A \implies \lnot B$ and $C \implies B$

    \begin{tabular}{ c | c || c}
    $A$ & $B$ & $A \implies \lnot B$\\
    \hline
    0 & 0 & 1\\
    0 & 1 & 1\\
    1 & 0 & 1\\
    1 & 1 & 0
    \end{tabular}\\ 
    \\
    \begin{tabular}{ c | c || c}
    $B$ & $C$ & $C \implies B$\\
    \hline
    0 & 0 & 1\\
    0 & 1 & 0\\
    1 & 0 & 1\\
    1 & 1 & 1
    \end{tabular}

    We can put these two truth tables together while conveying the same information

    \begin{tabular}{ c | c | c || c | c}
    $A$ & $B$ & $C$ & $A \implies \lnot B$ & $C \implies B$\\
    \hline
    0 & 0 & 0 & 1 & 1\\
    0 & 0 & 1 & 1 & 0\\
    0 & 1 & 0 & 1 & 1\\
    0 & 1 & 1 & 1 & 1\\
    1 & 0 & 0 & 1 & 1\\
    1 & 0 & 1 & 1 & 0\\
    1 & 1 & 0 & 0 & 1\\
    1 & 1 & 1 & 0 & 1
    \end{tabular}

    The truth table for $(A \implies \lnot B) \land (C \implies B)$ is only $1$ when {\it both} of the right columns are $1$, so it'll look like this

    \begin{tabular}{ c | c | c || c | c | c} 
    $A$ & $B$ & $C$ & $A \implies \lnot B$ & $C \implies B$ & $(A \implies \lnot B) \land (C \implies B)$\\
    \hline
    0 & 0 & 0 & 1 & 1 & 1\\
    0 & 0 & 1 & 1 & 0 & 0\\
    0 & 1 & 0 & 1 & 1 & 1\\
    0 & 1 & 1 & 1 & 1 & 1\\
    1 & 0 & 0 & 1 & 1 & 1\\
    1 & 0 & 1 & 1 & 0 & 0\\
    1 & 1 & 0 & 0 & 1 & 0\\
    1 & 1 & 1 & 0 & 1 & 0
    \end{tabular}

  \end{enumerate}
\end{Choices}

The following questions have a proposition and corresponding proof. For each question: decide whether the proof is correct, and if not, identify the proof's flaw.

% direct proofs ----------------------------
\item We call integer $n$ an even number if and only if there exists an integer $k$, such that $n = 2k$.\\
{\bf Proposition:} The negative of any even integer $n$ is even.\\
\textbf{Proof:} The proposition is true:
\begin{enumerate}[(1)]
\item By definition of even number, there exists some integer $k$, such that $n = 2k$.
\item Multiply both sides by $-1$, we get
  \begin{align*}
    -n =& -(2k)\\
    =& 2 \times (-k)
  \end{align*}
\item Now let $r = -k$. We have $-n = 2r$ for some integer $r$.
\item Hence, by definition of even number, $-n$ is even. 
\end{enumerate}
\begin{enumerate}
\begin{Choices}
\TrueChoice\item The proof is correct.
\FalseChoice\item There is an error in line (1).
\FalseChoice\item There is an error in line (2).
\FalseChoice\item There is an error in line (3).
\FalseChoice\item There is an error in line (4).
\Solution This proof is correct; it is a form of direct proof.
\end{Choices}
\end{enumerate}
% http://www.mathcs.bethel.edu/~gossett/DiscreteMathWithProof/CommonErrorsInProofs.pdf
% counterexample problems -------------------
\item {\bf Proposition:} For any positive integer $k$, if $2^k = 0 \mod(3)$ then $8^k = 1 \mod{3}$.\\
\textbf{Proof:} The proposition is false. Proof by counterexample: let $k = 1$, then $8^1 = 8 \neq 1 \mod{3}$.
\begin{enumerate}
\begin{Choices}
  \FalseChoice\item The proof is correct. 
  \FalseChoice\item The proof is incorrect because the proof only
  mentions the case of $k = 1$, while the original proposition
  contains a universal quantifier for any positive integer $k$.
  \FalseChoice\item The proof is incorrect because for any positive 
  integer $k$, $2^k = 0 \mod{3}$ will always be false.
  \TrueChoice\item The proof is incorrect because the `counterexample' 
  provided is not a counterexample for the overall proposition. 
  Letting $k = 1$ does not make the implication false.
  \Solution The proof is incorrect because the provided counterexample is not actually a counterexample. The hypothesis of the proposition is always false, so the implication is always true. Thus, the statement is vacuously true and cannot be a counterexample.
\end{Choices}
\end{enumerate}
\item {\bf Proposition:} For any integer $n$, if $n^2 = 0 \mod(4)$ then $n = 0 \mod{4}$.\\
\textbf{Proof:} The proposition is false. Proof by counterexample: let $n = 6$, then $n^2 = 36 = 0 \mod{4}$, but $n \neq 0 \mod{4}$.
\begin{enumerate}
\begin{Choices}
  \TrueChoice\item The proof is correct. 
  \FalseChoice\item The proof is incorrect because the proof only
  mentions the case of $n = 6$, while the original proposition
  contains a universal quantifier for any positive integer $n$.
  \FalseChoice\item The proof is incorrect because the proposition is actually true.
  \Solution The proof is correct; it is an example of proof by contradiction.
\end{Choices}
\end{enumerate}
\item \textbf{Proposition: }Let $x$ and $y$ be two positive integers. 
  If $x \times y < 36$ then $x < 6$ or $y < 6$.\\
  \textbf{Proof: } 
  \begin{enumerate}[(1)]
  \item Suppose $x \geq 6$ and $y \geq 6$. 
  \item Since $x \geq 6$, multiply both sides by $y$, we get
    \begin{align*}
      x \times y \geq 6 \times y
    \end{align*}
  \item and since $y \geq 6$, multiply both sides by $6$, we get
    \begin{align*}
      6 \times y \geq 6 \times 6
      \text{ i.e. } 6 \times y \geq 36
    \end{align*}
  \item So $x \times y \geq 36$
  \item Therefore, if $x \times y < 36$ then $x < 6$ or $y < 6$.
  \end{enumerate}
  \begin{Choices}
    \begin{enumerate}
    \TrueChoice\item The proof is correct.
    \FalseChoice\item The proof is incorrect because (1) is not a negation of the
      original implication's conclusion.
    \FalseChoice\item The proof is incorrect because there is an error at (2).
    \FalseChoice\item The proof is incorrect because from (1) - (4) we cannot get (5).
    \end{enumerate}
    \Solution The proof is correct. It is a proof by contraposition, 
    i.e., to prove ``$x \times y < 36 \implies (x < 6 \lor y < 6)$'', 
    we first prove ``$(x \geq 6 \land y \geq 6) \implies x \times y \geq 36$''
  \end{Choices}
\item \textbf{Proposition: } The negative of any irrational number is irrational.\\
  \textbf{Proof: } 
  \begin{enumerate}[(1)]
  \item Assume (for the sake of contradiction) that there exists an
    irrational number $x$, such that $-x$ is rational.
  \item By definition of rational, $\exists a,b \in \mathbb{Z}$, $b\neq 0$, s.t.
    \begin{align*}
      -x = a/b
    \end{align*}
  \item multiply both sides by $-1$, we get
    \begin{align*}
      x = -(a/b) = (-a)/b
    \end{align*}
  \item Since $-a$ and $b$ are integers and $b \neq 0$, and by
    definition of rational, $x$ is rational, which is a contradiction.
  \item Therefore, the negative of any irrational number is irrational.
  \end{enumerate}
  \begin{Choices}
    \begin{enumerate}
    \TrueChoice\item The proof is correct.
    \FalseChoice\item The proof is incorrect because (1) is not a negation of the
      original proposition.
    \FalseChoice\item The proof is incorrect because there is an error at (4).
    \FalseChoice\item The proof is incorrect because from (1) - (4) we cannot get (5).
    \end{enumerate}
    \Solution The proof is correct. It is a proof by contradiction, 
    i.e., to prove ``$\forall$ irrational number $x$,$-x$ is irrational.''
    we show that ``$\exists$ irrational number $x$, $-x$ is rational.''
    leads to a contradiction.
  \end{Choices}

\end{enumerate}

\end{document}
